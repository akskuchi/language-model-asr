\documentclass[a4paper]{article}

\usepackage[english]{babel}
\usepackage[utf8]{inputenc}
\usepackage{amsmath}
\usepackage{graphicx}
\usepackage[colorinlistoftodos]{todonotes}

\title{Neural Network Language Models for Automatic Speech Recognition}

\author{Aditya Kaushik, Eduardo Rosado, Thomas Spilsbury}

\date{\today}

\begin{document}
\maketitle

\begin{abstract}
In this paper we compare Recurrent Neural Network (RNN)
approximations of language models to traditional n-gram based models.
We evaluate different recurrent model architectures, hyperparameter configurations,
encoder and decode configurations and regularization and intialization procedures
and compare their performance on various performance and natural language
generation metrics. We find that in general, Long Short-Term Memory models
with embedding layers at the encoder stage outperforms other models
in terms of inference metrics and computational performance.
\end{abstract}

\section{Introduction}
\label{sec:introduction}

One important problem faced by Automatic Speech Recognition (ASR)
systems is transcribing utterances by speakers into intelligible written language.
In general, this is a difficult problem to solve due to the varying length
of utterances and ambiguity of classification between one utterance for another. The result
is that the recognized transcript is likely
to contain many words that are not
intelligible, even if the transcribed words
are the system's most likely direct
transcription from sounds to syllables.

As an alternative, we can make use of prior
information about the language to select
words from a vocabulary given some
observations of sounds or characters and the words
that have been predicted beforehand. The language model,
then, uses information about the sequence of prior
predicted words to give a probability distribution
for the next word, such that the ASR system would
pick the next word based both on what are the most
likely following words given a sequence as well
as the utterance that is made by the speaker.

Prior to the introduction of RNNs, most language models used n-grams
in Hidden Markov Model chains, where it was assumed that the $k$th
word depended only on the prior $n$ words. One key problem
with this model is that the context window is of a fixed size, leading
to a horizon problem. The Horizon problem occur because a if there were a word
in the word sequence which is extremely predictive of the $(n + 1)$th
word away from the current word, that would would simply not be taken
into account in that model. Worse still, the fixed size context window
means that unnecessary weight is put on words that happen to be in the context
window that may actually not be all that predictive were the context window
infinite.

In contrast, RNN's seek to solve this problem by introducing a Neural Network
architecture with recurrent connections, meaning that as the RNN makes predictions
over a sequence observations, it takes into account a hidden layer produced
by the previous observation (which in turn was a product of all the observations
before that). In effect, the RNN architecture encodes the contextual history
into the hidden layer vector allowing for an encoded approximation of infinite
contextual history.

Since the publication of \cite{Milkolov10}, new RNN approaches have shown
improvement on the state-of-the-art. We examine these new architectures and
techniques for future work discussed in the Milkolov paper and analyze whether their
results indicate that improvements can be made on the baseline set by the paper.

\section{Data}
\label{sec:data}

The first dataset we use is the WikiText dataset. This is a smaller dataset than
Gigaword and WSJ'92, but allowed us to
iterate faster and run more experiments.

This dataset is based on articles from
Wikipedia, containing 2,088,628 tokens
in the training set, 245,569 tokens in
the test set and 217,646 tokens in the
validation set.

Samples from the Dataset include:

"As with previous <unk> Chronicles games , Valkyria Chronicles III is a …"

"At Nintendo CEO Hiroshi Yamauchi 's request , Game Boy creator Gunpei Yokoi 's Nintendo R & D1 developed …"

"It was larger than the Scientific , at 73 by 155 by 34 millimetres ..."

This dataset contained quite a lot of
technical or domain specific
terminology due to the fact that the
sentences within were derived from
encyclopedic articles. This inherently
makes accurate prediction difficult, since
there is a higher likelihood that the
following word might be infrequent in the
corpus.

\section{Metrics}
\label{sec:metrics}

Models are evaluated by taking a sequence
of words from the validation set, removing
$k$ words from the end of the sentence and predicting
the next $k$ words from the existing $n - k$ word sequence.

\subsection{Perplexity}
\label{sec:perplexity}

Perplexity is a measure of "how generalized" the language
model is. A higher perplexity indicates a greater uncertainty
was encountered when predicting sequences of words. Formally,
it is the inverse probability of a text sequence normalized
by the number of tokens.

$$ PP(W) = \sqrt{\prod \frac{1}{P(w|w_1, ..., w_{i - 1}}} $$

Note that perplexity is an "intrinsic metric", it does not
have anything to do with the quality of the predicted
sentences, but instead is based on the certainty of the
model itself.

\subsection{ROUGE}
\label{sec:rouge}

In contrast, metrics such as ROUGE (which is an improvement
on BLEU) measure the quality of a sentence in terms of
their similarity to human generated language. These metrics
are called "extrinsic metrics".

The most basic form of such extrinsic metric is "word error
rate" (WER), defined as:

$$ WER = \frac{S + D + I}{N} $$, where $N$ is the number of
words and $S, D, I$ are substituions, deletions, insertions.

However, such a metric doesn't take int account the fact that
a recognized sequence may differ in length to a reference
sequence.

Improving on this situation, ROUGE measures based on the
longest matching subsequences, overlapping pairs and n-gram
co-occurrences.

\subsection{Naive Accuracy}
Accuracy is measured naively by comparing
a validation set sentence to a predicted sentence and
seeing how many words the model predicted correctly.

\section{Models}
\label{sec:models}

\subsection{RNN}
\label{sec:lstm}

A simple RNN (Elman network) as described in the Milkolov
paper consists of a single linear layer and non-linear
activation (in our case, the $tanh$ function, $ \frac{e^{2x} - 1}{e^{2x} + 1} $, with domain $[ -\infty, \infty ]$ and
range $ [-1, 1] $. The inputs to the network are
a hidden state vector, $h_{t}$, where the size is a hyperparameter
and the input encoding vector $i_{t}$. The output of the network
is the updated hidden state vector $h_{t + 1}$, which will will
be passed as the hidden state vector when processing the $i_{t + 1}$th
word, and the output encoding vector $o_{t}$, which can be
viewed as a discrete probability distribution for the predicted word
at $t$ by applying the softmax function $ \frac{e^x}{\sum e^x_i} $.

The trainable parameters of the RNN are the weights of the linear
layer which determine how information from the previous state of the
hidden state vector and the incoming word vector encoding are encoded
into both the updated hidden state vector and output word vector encoding.

The RNN is trained via "back-propagation through time" (BPTT), which is
effectively the same as the back-propgation algorithm proposed for feed-forward
neural networks, but applied to the recurrent structure of the RNN. In effect,
BPTT involves the same application of the chain rule to recurrent function
applications, and hidden layer states, but usually with some sort of threshold
to prevent computational complexity scaling with the number of tokens the
network has seen so far. In practice this means that the hidden layer encodes
a context window \emph{up to} to a certain length, but no longer than
that length.

\subsection{LSTM}
\label{sec:lstm}

\subsection{GRU}
\label{sec:gru}

\section{Encoding and Decoding}
\label{sec:encdec}

\subsection{Embeddings}
\label{sec:embedddings}

\subsection{Shortlists}
\label{sec:shortlists}

\subsection{Tied Weights}
\label{sec:tiedweights}

\section{Regularization}
\label{sec:regularization}

\subsection{Dropout}
\label{sec:dropout}

\section{Initialization}
\label{sec:initialization}

\subsection{Xavier}
\label{sec:xavier}

\section{Model Configurations}
\label{sec:configuration}

\section{Experimental Results}
\label{sec:results}

\section{Analysis}
\label{sec:analysis}

\section{Future Work}
\label{sec:future}

\begin{thebibliography}{9}
\bibitem{nano3}
  K. Grove-Rasmussen og Jesper Nygård,
  \emph{Kvantefænomener i Nanosystemer}.
  Niels Bohr Institute \& Nano-Science Center, Københavns Universitet

\end{thebibliography}
\end{document}
